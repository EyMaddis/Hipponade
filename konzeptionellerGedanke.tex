\documentclass[12pt,a4paper,oneside,ngerman]{article}

\usepackage[utf8]{inputenc}  
\usepackage[ngermanb]{babel}
\usepackage{url} 

\begin{document}

\title{Hipponade - Eine moderne Corporate Identity} % Titel ausdenkan
\author{Maria-Anna Kandsorra, Mathis Neumann, Nelson Tavares de Sousa}
\maketitle
\newpage

\tableofcontents

%
%
%  Kein "man" benutzen... bspw. "wie Blabla, wenn man..."
%
%

\newpage

\section{Idee} % Idee
Dieses Semester sollten wir uns mit der Erstellung einer Identity-Site beschäftigen.
Im Rahmen dessen legten wir zu Beginn fest, uns auf eine Corporate-Identity zu spezialisieren.

Gemeinsam haben wir uns für einen fiktiven Getränkehersteller entschieden. Es sollen das Unternehmen und dessen Produkte präsentiert werden, aber es soll im Sinne des Marketings auch der Lifestyle der Zielgruppe bedient werden. 
Wir beschränken uns auf die Produktpräsentation und weitere Inhalte und Angebote, einen Onlinevertrieb realisieren wir nicht.

 Zu beachten gilt es hierbei, mehrere mögliche Personenrollen möglichst optimal anzusprechen:
\begin{itemize}
\item (Potentieller) Kunde/Konsument

Personen dieser Kategorie suchen nach Informationen zu Produkten und dem Unternehmen. 
Zwar sehen wir unsere Zielgruppe zwischen 18 und 30 Jahren, die technologisch erfahren ist, aber um die einfachste Nutzungserfahrung für den Kunden zu schaffen, können keine allgemeine Aussagen getroffen werden, wie erfahren die Personen im Umgang mit Internetseiten sind. 
Logische Folgerung dessen ist, diesem Personenkreis möglichst einfach die Informationen zugänglich zu machen. Außerdem sollten möglichst große bunte Produktbilder die Seiten verzieren. Ein weiterer interessanter Punkt ist die Kundenbindung, welche später genauer angezeigt wird. 

\item Händler

Händler besuchen die Seite, um sich über das Unternehmen zu informieren und sich eventuell als solchen anzubieten. Händler besitzen größere Erfahrung und suchen eher gezielt nach Informationen. Direkter Zugang zu Kontaktmöglichkeiten und/oder Funktionen für Händler sind hier in unseren Augen sinnvoll. Ebenso wie ein gutes Image via Geschichte und Nachhaltigkeit unserer Produktpalette.

\item Administrator

Diese Personen kennen die Seite und besitzen ausreichend Kenntnisse, wie diese anzuwenden ist.
Umfangreiche konzeptionelle Maßnahmen müssen hier nicht in dem Rahmen umgesetzt werden, wie in den oben genannten Gruppen.

\end{itemize}
Somit haben wir einen Rahmen geschaffen, indem wir uns bei der Realisierung bewegen können. Um den Fokus auf die Identity zu setzen, liegt besonderer Fokus auf dem Kunden.

\subsection{Angestrebte Corporate Identity}
Mit unserer fiktiven Firma 'Hipponade' streben wir nach einer modernen Identity, mit dem Hauptfokus auf die junge, an Nachhaltigkeit und Gesundheit denkende Generation abseits des Mainstreams.
Hipponade ist der Getränkehersteller für den modernen Hipster~\footnote{siehe \url{http://de.wikipedia.org/wiki/Hipster_(21._Jahrhundert)}}, bei dem das öffentliche Trinken einer Hipponade Flasche klarstellt, dass man ein gesund lebender Mensch ist, welcher durch den Verzehr unserer Produkte Umweltschäden unterbinden und Unternehmen die für Nachhaltikeit stehen unterstützen wollen. Dabei spielt das Fairtrade- wie auch das Biosiegel eine Rolle.


\section{Usabilty Analyse} % Usability Analyse
Nachdem wir uns festgelegt haben, was für ein Typ von Identity wir präsentieren wollen, haben wir 'Konkurrenten' ausgewählt, um deren Internetpräsenzen zu analysieren. 

In den folgenden Unterabschnitten geben wir kurz wichtige Punkte der Analyse zu jeder Seite wieder. Die entsprechenden Tabellen finden Sie im PDF-Dokument. % PDF erstellen und Namen nennen

\subsection{Absolut Vodka - www.absolut.com} % Absolut
Am meisten sticht auf dieser Seite die Navigation hervor. Diese hat eine geringe Anzahl an Ebenen und überzeugt durch präzise Stichworte. Des Weiteren beschränkt sich das Unternehmen nicht auf die Präsentation der eigenen Produkte, sondern bietet auch Rezepte an und Informationen zu Veranstaltungen. Dies übertrifft die Erwartungshaltung des Besuchers, der lediglich Informationen zu Produkten erwartet. Eine gesteigerte Kundenbindung ist hier zu erwarten. Diesen Punkt werden wir in das Konzept unserer Seite einfließen lassen.

Es werden hier gezielt nicht sofort Informationen zu den Produkten angeboten. Der Besucher soll nicht die Seite öffnen, gewünschte Informationen finden und dann diese wieder verlassen. Vielmehr wird hier versucht, den Besucher gezielt erst Informationen rund um das Unternehmen zu präsentieren. Dies fällt besonders in der Struktur der Navigationsleiste auf, in der die Getränke als letzter Punkt aufgeführt sind. Da es heutzutage in der Natur des Menschen liegt, von links nach rechts Informationen zu sammeln, werden ihm hier zuerst die Informationen zu Rezepten und Veranstaltungen auffallen. Dies trifft natürlich nur auf die europäische Kultur zu, was jedoch bei uns keine Bedeutung findet, da wir uns lediglich auf den deutschen Raum beschränken.

\subsection{Bacardi - www.bacardi.de} % Bacardi
Hier fällt, wie bereits bei Absolut, die einfache Navigation auf. Die geringe Anzahl an Ebenen vereinfacht das Zurechtfinden auf der Seite. Es wird eine Suche angeboten, die von jeder Seite erreichbar ist. Negativ fällt hier auf, dass die Seite an einigen Stellen "überladen" wirkt. Videos und zu viele Designelemente führen zu einer trägen Informationswahrnehmung. Hier läuft Bacardi Gefahr Kunden zu "langweilen" oder zu verschrecken. Dies gilt es für uns zu vermeiden.

Auch hier fällt wieder auf, dass die wichtigste Information für den Besuche gezielt nachrangig angeboten wird. So finden wir hier, wie bereits bei Absolut Vodka, die Seite zu den Getränken erst an letzter Stelle der Navigationsleiste.

\subsection{Fritz Kola - www.fritz-kola.de} % Fritz
Der Webauftritt von Fritz Kola hat ein nicht sehr homogenes Design. Einige Seiten sind sehr schlicht und bieten ein sehr klassisches Layout. 
Hingegen nutzen andere Seiten, wie z.B. die Produktseiten, modernere Webtechnologien und wirken allgemein modern.
Die Seite macht allgemein einen nicht ganz durchdachten Eindruck, z.B. bietet das Info-Karussell keine textuellen Überschriften, des weiteren sind einige tote Verlinkungen zu finden, sowie visuelle Designfehler vorhanden.
Es wird zwar eine Sprachauswahl angeboten, aber die wird nahezu überall missachtet.
Diese Fehler gilt es für uns zu vermeiden und stattdessen auf eine simple und übersichtlichere Struktur zu setzen.
Fritz Kola hat einige weiterführende Inhalte zu bieten, die wahrscheinlich redaktionell gepflegt werden und so zum erneuten Besuch motivieren.
Der klare Fokus auf die Zielgruppe von jungen 'Partygehern' ist auffällig und wir werden versuchen dies auch so konsequent für unsere Zielgruppe durchzuführen.

\subsection{Mate Drinks - www.matedrinks.de} % Mate
Die Club-Mate ist der inbegriff eines 'Hipster'-Getränks. Allerdings lässt der Internetauftritt zu wünschen übrig. Informationen über die Geschichte und deren Zutaten, außerdem weiteren Getränken der Marke sind vorhanden. Das Image des Unternehmen besteht nur aus Tradition, es wird wenig Wert auf Nachhaltigkeit oder auf Biosiegel gelegt. Das Design ist einfach und plump. Es wird keine Beziehung zu dem Kunden aufgebaut. Allerdings gibt es Fremdsprachenunterstützung für Englisch, Französisch und Niederländisch. Das finden wir gut, aus Zeitmangel wird es aber nicht übernommen. Die Webseite scheint veraltet und die letzten Neuigkeiten sind ein halbes Jahr alt, das möchten wir auf keinen Fall übernehmen. Die Vielfältigkeit der Produktpalette überrascht und einfache Informationen werden weitergegegben, dieses Ziel wollen wir teilweise verfolgen. Unsere Palette wollen wir überraschend gestalten, aber den Informationsgehalt ein wenig intensivieren.

\subsection{MyMuesli - www.mymuesli.de} % MyMuesli
MyMueli wird vermutlich redaktionell gepflegt und erhält des Öfteren Aktualisierungen in mehreren Bereichen und auch sogar Sprachen. 
Das Navigationsprinzip ist nicht seitenübergreifend und in Verbindung mit der Typographie dauert das Finden von speziellen Inhalten länger als erwartet.
Sogar noch stärker als Fritz Kola versucht MyMuesli den Besucher/Kunden durch Vielfalt und Aktualität zu binden.
Wir werden ebenfall versuchen das Angebot möglichst vielfältig zu gestalten und dabei den Besucher nicht zu überfordern.
Weiter legt MyMuesli einen Priorität auf die Darstellung der Qualitäten wie Nachhaltigkeit, Bioprodukte und Gesundheitsaspekte in Verbindung mit den Produkten, ein Ziel was wir ebenfalls verfolgen.

\subsection{Alnatura - www.alnatura.de} %Alnatura
Alnaturas INernetauftritt ist bunt und Informationsvoll. Es gibt sehr viele Menüpunkte, auch viele die nichts mit den Produkten zu tun haben sondern welche das Image des Unternehmen bezüglich Natur, Nachhaltigkeit und Bio unterstützt. Dadurch baut der Kunde eine gute Beziehung mit dem Unternehmen auf. Die Zielgruppe scheint gesundheits- und naturbewusste Menschen zu sein, insbesondere Familien. Es werden unter anderem aktuelle Angebote angegeben, ebenso wie eine reiche Vielfalt von Rezepten, welche nicht unbedingt auf ihre eigenen Produkte beschränkt sind. Die Produktpalette ist einem Supermarkt entsprechend groß und es gibt eine Vielzahl an Informationen bezüglich Ernährung, insbesondere Babynahrung, die man als Kunde auf der Website nicht erwartet aber in das Schema passen. Die Informationsvielfalt ist zusammenfassend gesagt, sehr groß. Wir finden sie aber zu groß für unser Unternehmen, da wir nicht eine so große Produktpalette wie Alnatura besitzen. Das Design scheint aber nicht ganz durchdacht. Sobald man besagte Informationen abfragen möchte, übernimmt die seitliche Menüleiste einen drittel des Bildschirms. Das wollen wir auf keinen Fall übernehmen. Man scheint dadurch besser navigieren zu können, aber dies ist wohl für eine ältere Kundgruppe gedacht als diese, die wir ansprechen wollen. Alnatura beschränkt sich bei ihren Neuigkeiten auf Presseartikel, in denen sie selbst vorkommen. Die Website wird laut Website redaktionell gepflegt.

\subsection{Zusammenfassung}
Anhand der Analyse sind uns somit Punkte aufgefallen, die vermieden werden sollten. Einige Aspekte fallen jedoch sehr positiv aus. Diese sind solche, welche die Erwartungshaltung der Besucher übertreffen können oder direkten Einfluss auf die Wahrnehmung des Besucher bewirken.


\begin{itemize}
\item{Positive Punkte}	
\begin{itemize} % Positive Punkte
	\item{Geringe Anzahl an Navigationsebenen}
	\item{Präzise Stichwörter für die Navigation}
	\item{Fixe Navigationsleiste}
	\item{Suche permanent erreichbar}
	\item{Rezepte}
	\item{Veranstaltungen}
	\item{Geschichte des Unternehmens}
	\item{Klarer Fokus auf Zielgruppeninteresse}
	\item{Indirekte Informationsführung}
	\item{Jedes Produkt hat individuelle Design Besonderheiten}
\end{itemize}
\end{itemize}


\begin{itemize}
\item{Negative Punkte}	
\begin{itemize} % Negative Punkte
	\item{Überladene Seiten}
	\item{Nicht homogenes Design}
	\item{Plumpes Design}
	\item{Große seitliche Menüleiste}
\end{itemize}
\end{itemize}

Die positiven Punkte sind unserer Meinung nach, insbesondere im Hinblick auf die Präsentation der Identity, von Bedeutung. Es ist zu erkennen, dass momentan ein Trend verfolgt wird, dem Besucher nicht nur die Produkte nahe zu bringen. Vielmehr gilt es das Unternehmen und dessen Image zu vermarkten. 

Durch Punkte, wie Rezepte oder Geschichte, wird zu dem Kunden eine emotionale Bindung aufgebaut, um diesen langfristig für sich zu gewinnen. 
Es wird eine Identity erzeugt, die den persönlichen Interessen des Kunden ähnelt.
Da dies ein sehr interessanter und effektiver Aspekt ist, werden wir dies auch anwenden.

\section{Medienobjekte} % Medienobjekte
Aus der Analyse und unserem Rahmen folgen bestimmte Objekte die es abzubilden gilt. Das offensichtlichste sind bspw. die Produkte, aber natürlich Fallen auch weitere an.
\begin{itemize}
\item Bild

Auf der Seite sollen Bilder dargestellt werden und alle weiteren Medienobjekte visuell erweitern. Diese können von einem Admin organisiert werden.

\item Rezept

Um den Besucher zum wiederholten Aufrufen zu motivieren und die Kundenbindung zu stärken, werden Rezepte angeboten, welche unsere Produkte in irgendeiner Form mit einbeziehen.
Fokus liegt hierbei noch auf die Nährwerte des Rezept als Teil unserer gesunden Identity.
Plan ist diese Rezepte redaktionell pflegen zu lassen, was auch die Nährwertangaben möglich macht.

\item Zutat

Zutaten werden mit Rezepten und Produkten in Verbindung verbracht. Dies soll bspw. das Filtern nach bestimmten Zutaten ermöglichen (nicht implementiert, siehe unten).

\item Verkaufsstellen/Händler

Da ein so kleiner 'Szene-Getränke'-Hersteller eventuell nicht in jedem Supermarkt zu finden ist, möchten wir dem Besucher die Möglichkeit bieten schnell Geschäfte zu finden, die Hipponade Getränke verkaufen. 
Dabei ist vor allem der Ort wichtig, welcher auch auf einer Interaktiven Karte dargestellt werden soll.
Händler haben hier die Möglichkeit selbst solche einzutragen, geben ihre Adresse an und wir berechnen dann daraus automatisch Koordinatendaten für die Kartendarstellung.

\item Tag

Tags sind kurze Schlagworte, zur Gruppierung von Inhalten.
Produkte, Rezepte, sowie News und Veranstaltungen werden mit Tags versehen. Dies ermöglicht die Realisierung einer Suchfunktion auf der Seite. 
Darüber hinaus sind sie dem versiertem Social Media Nutzer, was unsere Zielgruppe größtenteils ist, als 'Hashtags' (mit anführendem \#) bekannt, wie sie bei Facebook, Twitter, etc. genutzt werden. 
Sie erlauben einfache und schnelle Gruppierung, ohne das eine aufwendige Kategorie-Struktur implementiert werden muss, da dies mit dem Vorsatz der flachen Navigation brechen würde.

\item Veranstaltung

Um sich als 'Szene-Getränk' zu etablieren müssen wir dort agieren, wo wir den Großteil unserer Zielgruppe vermuten.
Dies sind vor allem Veranstaltungen, wie Partys und Konzerte, die wir einerseits organisieren möchten, aber auch durch Sponsoring als Marketing Plattform nutzen.

\item Produkt

Herzstück der Firma sind natürlich die Getränke.
Hierbei legen wir im Zuge unserer Identity Fokus auf die Darstellung der Qualitäts und Gesundheitsaspekte unserer Getränke. 
Wir stellen also Nährwertangaben, Zutaten, sowie Eigenschaften wie 'Bio', 'vegan', 'Fair Trade' etc. hervor.
Letzteres machen wir indirekt über die Verwendung entsprechender Tags.
Um die Individualität der Produkte hervorzuheben, bieten wir die Möglichkeit 

\item Nährwerte

Nährwerte werden in Produkten und Rezepten zugeordnet. Dies dient dem erweiterten Informationsumfang den wir unseren gesundheitsbewussten Kunden anbieten wollen.
Auf lange Sicht wäre eine Filterungen, z.B. alle Rezepte bis zu 400kcal, zu implementieren.

\item News

Wir haben die Aktualität der Seiten als einen zentralen Aspekt bei der Usability von Webseiten in der Zielbranche festgestellt.
Aus diesem Grund bieten wir ebenfalls eine Neuigkeiten Seite, die regelmäßig redaktionell gepflegte Inhalte bieten soll.

\item Benutzer \& Rollen

Dient dem Sicherheitskonzept, bei dem ein Benutzer nur Aufgaben erfüllt, zu denen seine Rolle Berechtigungen hat. 
Aktuell beschränken wir uns auf eine einzige Administrator Rolle.
Ein Registrierungsprozess für Nicht-Mitarbeiter der fiktiven Firma ist nicht vorgesehen.

\end{itemize}


\section{Story-Diagramm} % Story-Diagramm

Das Story-Diagramm erläutert bildlich die Vorgänge, welche die Medienobjekte durchgehen.
Diese können nur vom Administrator durchgeführt werden. Lediglich Händler können Stores hinzufügen.
Um die Übersicht zu verbessern wurde das Diagramm aufgeteilt. Bubbles ohne sub-unit o.ä. dienen an der jeweiligen Stelle als Platzhalter. Die richtigen Spezifikationen hierzu, sind an anderer Stelle zu entnehmen. Das Diagramm finden Sie unter Diagramm.pdf.


\section{Sonstige Besonderheiten der Implementation}

\begin{itemize}
	\item Wir bieten in den Neuigkeiten und Veranstaltungen die Möglichkeit für die Redaktion Markdown~\footnote{\url{http://de.wikipedia.org/wiki/Markdown}} für das Text Layout zu verwenden
	\item Beim jeglichen administrativen Arbeiten haben wir auf Bedienkomfortfunktionen verzichtet unser Fokus ist ganz klar auf dem was der Endkunde/-Benutzer zu sehen bekommt.
	
\end{itemize}

\section{Anhang}

Unsere weiteren Anhänge sind als einzelne Dateien vorzufinden.

\end{document}
