\documentclass[12pt,a4paper,oneside,ngerman]{article}

\usepackage[utf8]{inputenc}  
\usepackage[ngermanb]{babel} 

\begin{document}

\title{Konzeptioneller Gedanke} % Titel ausdenkan
\author{Maria Kandsorra, Mathis Neumann, Nelson Tavares de Sousa}
\maketitle
\newpage

\tableofcontents

%
%
%  Kein "man" benutzen... bspw. "wie Blabla, wenn man..."
%
%

\newpage

\section{Idee} % Idee
Dieses Semester sollten wir uns mit der Erstellung einer Identity-Site beschäftigen.
Im Rahmen dessen legten wir uns seit Beginn an fest, eine Corporate-Identity zu realisieren.

Infolgedessen entschieden wir uns für einen fiktiven Getränkehersteller. Es sollen das Unternehmen, sowie dessen Erzeugnisse präsentiert werden. Wir beschränken uns lediglich auf die Produktpräsentation. Ein Onlinevertrieb realisieren wir nicht.

 Zu beachten gilt es hierbei, mehrere mögliche Personenrollen möglichst optimal anzusprechen:
\begin{itemize}
\item (Potentieller) Kunde/Konsument

Personen dieser Kategorie suchen nach Informationen zu Produkten und das Unternehmen. Es können keine allgemeine Aussagen getroffen werden, wie erfahren die Personen im Umgang mit Internetseiten sind. Logische Folgerung dessen ist, diesem Personenkreis möglichst einfach die Informationen zugänglich zu machen. Ein weiterer Interessanter Punkt ist die Kundenbindung, welche später genauer angezeigt wird.
\item Großhändler

Großhändler besuchen die Seite, um sich über das Unternehmen zu informieren und sich eventuell als solchen anzubieten. Großhändler besitzen größere Erfahrung und suchen eher gezielt nach Informationen. Direkter Zugang zu Kontaktmöglichkeiten und/oder Funktionen für Großhändler sind hier in unseren Augen sinnvoll.
\item Administrator

Diese Personen kennen die Seite und besitzen ausreichend Kenntnisse, wie diese anzuwenden ist.
Umfangreiche konzeptionelle Maßnahmen müssen hier nicht in dem Rahmen umgesetzt werden, wie in den oben genannten Gruppen.
\end{itemize}
Somit haben wir einen Rahmen geschaffen, indem wir uns bei der Realisierung bewegen können.

\newpage % Ggf. entfernen
\section{Usabilty Analyse} % Usability Analyse
Nachdem wir uns festgelegt haben, was für ein Typ von Identity wir präsentieren wollen, haben wir "Konkurrenten" ausgewählt, um deren Internetpräsenzen zu analysieren. 

In den folgenden Unterabschnitten geben wir kurz wichtige Punkte der Analyse zu jeder Seite wieder. Die entsprechenden Tabellen finden Sie im PDF-Dokument. % PDF erstellen und Namen nennen

\subsection{Absolut Vodka - www.absolut.com} % Absolut
Am meisten sticht auf dieser Seite die Navigation hervor. Diese hat eine geringe Anzahl an Ebenen und überzeugt durch präzise Stichworte. Des Weiteren beschränkt sich das Unternehmen nicht auf die Präsentation der eigenen Produkte, sondern bietet auch Rezepte an und Informationen zu Veranstaltungen. Dies übertrifft die Erwartungshaltung des Besuchers, der lediglich Informationen zu Produkten erwartet. Eine gesteigerte Kundenbindung ist hier zu erwarten. Diesen Punkt werden wir in das Konzept unserer Seite einfließen lassen.
\subsection{Alnatura} % Alnatura

\subsection{Bacardi - www.bacardi.de} % Bacardi
Hier fällt, wie bereits bei Absolut, die einfache Navigation auf. Die geringe Anzahl an Ebenen vereinfacht das Zurechtfinden auf der Seite. Es wird eine Suche angeboten, die von jeder Seite erreichbar ist. Negativ fällt hier auf, dass die Seite an einigen Stellen "überladen" wirkt. Videos und zu viele Designelemente führen zu einer trägen Informationswahrnehmung. Hier läuft Bacardi Gefahr Kunden zu "langweilen" oder zu verschrecken. Dies gilt es für uns zu vermeiden.
\subsection{Fritz Kola} % Fritz

\subsection{Mate Drinks} % Mate

\subsection{MyMuesli} % MyMuesli

\subsection{Zusammenfassung}
Anhand der Analyse sind uns somit Punkte aufgefallen, die vermieden werden sollten. Einige Aspekte fallen jedoch sehr positiv aus. Diese sind solche, welche die Erwartungshaltung der Besucher übertreffen können oder direkten Einfluss auf die Wahrnehmung des Besucher bewirken.

Solche Punkte sind:
\begin{itemize}

\item{Positiv}
\begin{itemize} % Positive Punkte
\item{Geringe Anzahl an Navigationsebenen}
\item{Präzise Stichwörter für die Navigation}
\item{Fixe Navigationsleiste}
\item{Suche permanent erreichbar}
\item{Rezepte}
\item{Veranstaltungen}
\item{Geschichte des Unternehmens}
\end{itemize}

\item{Negativ}
\begin{itemize} % Negative Punkte
\item{Überladene Seiten}
\item{Nicht homogenes Design}
\end{itemize}

\end{itemize}

Die positiven Punkte sind unserer Meinung nach, insbesondere im Hinblick auf die Präsentation der Identity, von Bedeutung. Es ist zu erkennen, dass momentan ein Trend verfolgt wird, dem Besucher nicht nur die Produkte nahe zu bringen. Vielmehr gilt es das Unternehmen und dessen Image zu vermarkten. 

Durch Punkte, wie Rezepte oder Geschichte, wird zu dem Kunden eine emotionale Bindung aufgebaut, um diesen langfristig für sich zu gewinnen. Da dies ein sehr interessanter und effektiver Aspekt ist, werden wir dies auch anwenden.

\section{Medienobjekte} % Medienobjekte
Aus der Analyse und unserem Rahmen folgen bestimmte Objekte die es abzubilden gilt. Das offensichtlichste sind bspw. die Produkte, aber natürlich Fallen auch weitere an.
\begin{itemize}
\item Bild

Auf der Seite sollen Bilder dargestellt werden. Diese können von einem Admin organisiert werden.

\item Rezept

Für unsere Produkte werden Rezeptvorschläge angegeben.

\item Zutat

Zutaten werden mit Rezepten und Produkten in Verbindung verbracht. Dies Ermöglich bspw. das Filtern nach bestimmten Zutaten.

\item Verkaufsstellen

Verkaufsstellen zeigen an, wo bestimmte Produkte erworben werden können. Großhändler haben hier die Möglichkeit selbst solche einzutragen.

\item Tag

Produkte, Rezepte, sowie News und Veranstaltungen werden mit Tags versehen. Dies ermöglicht die Realisierung einer Suchfunktion auf der Seite.

\item Veranstaltung

Veranstaltungen gehören zu unserem Konzept. Diese dienen der Kundenbindung.

\item Produkt

Die Produkte des Unternehmens sollen auf der Seite präsentiert werden.

\item News

Ähnlich zu Veranstaltung gehört dies zum konzeptionellen Gedanken der Seite.

\item Benutzer

Dient dem Sicherheitssystem. Ein User verwaltet diese. Des Weiteren können lediglich User (Administratoren) Änderungen an der Seite vornehmen.

\item Nährwerte

Nährwerte werden Produkten und Rezepten zugeordnet. Dies dient dem erweiterten Informationsumfang den wir anbieten wollen.

\end{itemize}


\section{Story-Diagramm} % Story-Diagramm



\section{Modellierung der Datenbank} % DB

\section{Zusammenfassung} % Zusammenfassung
\end{document}
